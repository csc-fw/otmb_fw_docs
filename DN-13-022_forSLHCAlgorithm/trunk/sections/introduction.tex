\section{Introduction}

The muon trigger is a tracking trigger that measures the momentum of muons using the magnetic field in the steel yoke of the CMS detector~\cite{Chatrchyan:2008zzk}; thus its resolution degrades with increasing momentum. This can be improved by maximizing the number of chamber hits along a muon trajectory and by improving the precision and number of position and angular measurements participating in the track fit applied in the trigger logic. The focus of the muon trigger upgrade is to improve its rate reduction capability without significantly affecting the efficiency.

The philosophy applied to the design of the present muon trigger was to preserve the complementarity and redundancy of the three separate muon detection systems until they are combined at the input to the Global Trigger. In contrast, the upgrade to the muon trigger will utilize the redundancy of the three muon detection systems earlier in the trigger processing chain so as to obtain a more performant trigger with higher efficiency and better rate reduction. Recognizing that every additional hit along a muon trajectory further improves the fake rejection and muon momentum measurement, the upgrade seeks to combine muon hits at the input stage to the Muon Track-Finder layer rather than at its output, given the successful operation of all three muon detection systems in the LHC Run 1. This new Muon Track-Finder will ultimately replace the separate track-finders for the Drift-Tube (DT) and Cathode Strip Chamber (CSC) muon triggers as well as the Resistive Plate Chamber (RPC) pattern comparator trigger. However, in addition to combining data from multiple muon systems in the same processors, more robust and sophisticated algorithms will be applied that are tolerant of the increased pile-up and make better use of the data from each muon system in the track-finding and $p_T$ measurement. Present and upgraded muon trigger algorithms in CSC are described in Section~\ref{sec:present_algo} and Section~\ref{sec:SLHC_algo}, correspondingly. Individual improvements of the upgraded algorithm are studied in Section~\ref{sec:SLHC_algo_results}.

The muon trigger upgrade is not expected to be completed before the LHC resumes operation after Long Shutdown I in 2015. Thus it is imperative to be able to commission the new muon trigger in parallel with the operation of the current trigger. For that reason, the trigger primitives from the CSC and RPC systems will be fully split into two paths, and those from the DT system will be split from a slice of the system. Data concentration into high bandwidth links for all three systems should be available for at least a slice of the detector as input to the new trigger in 2015. The goal is to have the entire muon trigger upgrade commissioned during the 2015-16 Year-End Technical Stop, so that the new trigger will be available in 2016. Recognizing the physics importance of the first year of LHC running at a higher center-of-mass energy ($\sqrt{s} = 13$~TeV) in 2015, a provision is foreseen to apply isolation criteria to the CSC muons in the forward region of the detector using the energy deposits in the calorimeter. This will allow to maintain relatively low $p_T$ thresholds reducing the rate of muons from heavy flavor jets despite the higher collision energy and higher luminosity.

As a part of future upgrade for Phase II of LHC running, the CMS will be equiped with Gas Electron Multiplier (GEM) detectors in the high pseudorapidity region ($1.5 < |\eta| < 2.2$). Pairs of triple-GEM chambers will be installed in the currently vacant position in front of the ME1/1 chambers and are dubbed GE1/1. The addition of such chambers allows to measure the bending angle of a track between GE1/1 and ME1/1. Usage of the bending angle at L1 can help to keep the rates down while keeping the efficiency high. In section \ref{sec:slhc_algorithm_with_gems} we investigate several implementation possibilities of the combined GEM-CSC trigger algorithm for the high luminosity run of SLHC.